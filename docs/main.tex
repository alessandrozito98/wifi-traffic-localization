\documentclass[11pt, USenglish]{article}
\usepackage{graphicx}
\usepackage{booktabs,caption}
\usepackage{tabularx,booktabs}
\usepackage[flushleft]{threeparttable}
\usepackage[rightcaption]{sidecap}
\usepackage{wrapfig}
\usepackage{subcaption}
\usepackage[utf8]{inputenc}
\usepackage{setspace}
\usepackage{multirow}
\usepackage{amsmath}
\usepackage{bbm}
\usepackage{caption}
\usepackage{vmargin}
\usepackage[title]{appendix}
\usepackage{float}
\usepackage{listings}
\usepackage{caption}
\usepackage[square, numbers]{natbib}
\usepackage{tabularx}
\usepackage{lastpage}
\usepackage{comment}
\usepackage{hyperref}
%\usepackage{floatrow}
\usepackage{pgfplots}
\usepackage{emptypage}
\usepackage{lipsum}
\bibliographystyle{abbrvnat}
\usepackage[nottoc]{tocbibind}
\renewenvironment{abstract}
{\quotation\small\noindent\rule{\linewidth}{.5pt}\par\smallskip
{\centering\bfseries\abstractname\par}\medskip}
{\par\noindent\rule{\linewidth}{.5pt}\endquotation}
\providecommand{\keywords}[1]{\textbf{\textit{Keywords:}} #1}
\DeclareCaptionFormat{citation}{%
\ifx\captioncitation\relax\relax\else
\captioncitation\par
\fi
#1#2#3\par}
\newcommand*\setcaptioncitation[1]{\def\captioncitation{\textit{Source:}~#1}}
\let\captioncitation\relax
\captionsetup{format=citation,justification=centering}
\setpapersize{A4}
\setmargins{2.5cm}      
{1.5cm}               
{16.5cm}               
{23.42cm}        
{10pt}             
{1cm}                      
{0pt}                   
{2cm}                        

\renewcommand{\baselinestretch}{1.5}
\usepackage{fancyhdr}
\pagestyle{fancy}
\fancyhf{} 
\lhead{Wi-Fi encrypted traffic classification \hfill September 2022}
\usepackage{hyperref}
\rfoot{Page \thepage \hspace{1pt} of \pageref{LastPage}}
\usepackage{cleveref}
\setlength{\footnotesep}{\baselineskip}
\hypersetup{
colorlinks=true,
linkcolor=black,
filecolor=magenta, 
citecolor=blue,
urlcolor=cyan,
}
\usepackage{booktabs}

\usepackage{graphicx}
\graphicspath{ {./images/} }
\usepackage[utf8]{inputenc}

\title{Wi-Fi Encrypted Traffic Classification}
\author{Alessandro Zito, Carlos Santillán}
\date{September 6th 2022}



\begin{document}

\maketitle
% Insert the title, author and date
\begin{center}
\begin{tabular}{l r}
ID number: & 890219 - 102516\\ % Partner names
Personal code: & 10617579 - 10659783\\
Supervisor: & Alessandro Enrico Cesare Redondi \\% Instructor/supervisor
Course: & Wireless Internet
\end{tabular}
\end{center}


\newpage

\tableofcontents

\newpage


\section{Introduction}
This project aims to implement a classification algorithm in the context of Wi-Fi traffic. The classification problem in this context may aid in making more informed decisions in network resource allocation or when performing diagnostics.

The presented pipeline starts with sniffing encrypted Wi-Fi traffic in monitor mode, the subsequent construction of a traffic dataset, and the training of a classifier to distinguish different types of traffic. Finally, we will focus on a single known device, but the procedure can be generalized to many devices contemporaneously. Data analysis was performed with R.

\section{Sniffing and Dataset Building}

\subsection{Sniffing}


The first step in building the dataset is to capture Wi-Fi traffic. We decided to distinguish between five types of traffic:
\begin{itemize}
	\itemsep-0.3em 
	\item Idle device
	\item Web browsing
	\item VoIP calls 
	\item Video calls 
	\item Youtube streaming
\end{itemize}
We captured (in \textit{monitor mode}) approximately 20 minutes of traffic per type (via Wireshark) and stored them in .pcap files. We filtered traffic sent from and received by the target device\footnote{MAC address A4:42:3B:D2:F7:08}. Since the traffic is encrypted, we cannot rely on port numbers, IP addresses, etc., for the purposes of classification. Instead, we had to work exclusively with statistical features.

\subsection{Dataset Building}

Each entry in our dataset consists of a traffic flow lasting $W$ seconds. This required some preprocessing on the original data. For each traffic flow, which includes all packets sent and received by the target device in the interval $[t, \,t + W)$, we extracted the following statistical features:

\begin{itemize}
	\itemsep-0.3em 
	\item Mean and variance of packet length
	\item Mean and variance of inter-arrival time
	\item Maximum and minimum packet length
	\item Number of packets of type: QoS Data, QoS Null function, Other
	\item Number of sent and received packets
\end{itemize}


The choice of the parameter $W$ was based on the ability to distinguish the traffic types in that time frame significantly. Starting from 5 seconds and increasing, we finally chose 15 seconds since it met our criteria. So we processed the data to obtain the features and produced a dataset containing 477 rows.


\section{Data Exploration}

We looked at the variance of the features and noticed a significant unbalance (see Figure 1) that we addressed by normalizing the data (see Figure 2). 

\begin{figure}
	\begin{minipage}[c]{0.4\linewidth}
		\includegraphics[width=\linewidth]{boxplot.png}
		\caption{Boxplot of original features, variance of packet length creates an unbalance}
	\end{minipage}
	\hfill
	\begin{minipage}[c]{0.4\linewidth}
		\includegraphics[width=\linewidth]{boxplot2.png}
		\caption{Boxplot of normalized features, our classifier will no longer be biased towards one feature}
	\end{minipage}%
\end{figure}


Because of the high number of features, we performed principal component analysis for visualization purposes. The first three principal components explain over 75\% of the variability. We proceed to assess the separability of the data qualitatively. Figure 3 shows the PC1 v. PC2 plot. Figure 4 shows PC2 v. PC3, which yields similar results.
\begin{figure}
\centering
		\includegraphics[width=140mm, height=90mm,scale=0.5]{pc1pc2.png}
		\caption{The PC1 vs PC2 plot shows a significant degree of separability}
\vspace{8mm} 
	\centering
	\includegraphics[width=140mm, height=90mm,scale=0.5]{pc2pc3.png}
	\caption{Confirms a significant degree of separation in the PC2 v. PC3 plane}
\end{figure}

The plots confirm the presence of well-defined clusters. With this information, we proceed to train the classifier using the original features.

\section{Classification and results}
Many machine learning techniques can tackle this problem: logistic regression, support vector machines, Bayes classifiers, etc. However, we chose $k$-Nearest-Neighbours since the data was well-clustered for it to perform well, and it did not require any strict assumptions on normality and equal variance between clusters. When KNN classifies a new sample, it finds its $k$ nearest neighbors (in the training dataset) and classifies it in the most common class among those neighbors. We used the euclidean distance. We shuffled the data and divided it into a training set (65\%) and a test set (35\%) and tuned the hyper-parameter $k$ via leave-one-out cross-validation. 
\\\\
After finding 6 as our optimal $k$, we proceed to compute the test error\footnote{The test set is normalized using the same normalization obtained from the training set.}. We obtain an accuracy of 88.6\%. Figure 5 shows the confusion matrix, and Figure 6 approximates the classification region. We see that web browsing and idle traffic are difficult to distinguish, whereas video calls are much different from the others; therefore, it is easy to classify them correctly.
\begin{figure}
	\centering
	\includegraphics[width=120mm, height=70mm,scale=0.5]{confusion.png}
	\caption{Confusion matrix of the test set}
\end{figure}
\begin{figure}
	
	\centering
	\includegraphics[width=140mm, height=90mm,scale=0.5]{region.png}
	\caption{An approximation of the classification region using the first two principal components}
\end{figure}

\section{Conclusion}

We have reached the goal of classifying traffic with an acceptable level of accuracy. It may be possible to achieve even better results with a further transformation of the original features (use of basis functions) or via kernel functions and support vector machines. In addition, the pipeline can be smoothly adapted into a real-time environment so that we may monitor the state of the network (at least in terms of traffic) and allocate resources accordingly.
\newpage
	\begin{thebibliography}{99} 
	\bibitem{towards_deployment}
	Pacheco, Fannia, Ernesto Exposito, Mathieu Gineste, Cedric Baudoin, and Jose Aguilar. "Towards the deployment of machine learning solutions in network traffic classification: A systematic survey." IEEE Communications Surveys \& Tutorials 21, no. 2 (2018): 1988-2014.
	
	\bibitem{mla_classification}
	Li, Wei, and Andrew W. Moore. "A machine learning approach for efficient traffic classification." In 2007 15th International symposium on modeling, analysis, and simulation of computer and telecommunication systems, pp. 310-317. IEEE, 2007.
  
\end{thebibliography}


\end{document}
